\begin{abstract}
  Les approches par réseaux de neurones ont récemment surpassées les approches par algorithmes classiques en ce qui concerne la majorité des problèmes du traitement de la langue naturelle. C'est principalement ce qui explique pourquoi nous voyons de plus en plus de solutions commerciales implémentant des agents conversationnels, tels que \textbf{Siri (Apple)}, \textbf{Alexa (Amazon)} et \textbf{Google Assistant et Google Home (Google)}. \\

  Devons-nous toutefois croire que la réalisation d'un tel agent et une science réservée aux géants de l'industrie technologique? Comment s'y prennent-ils? Est-ce envisageable de se créer notre propre agent conversationnel? Si oui, Comment? \\
  %[sources? Et est-ce que google assistant ca ce nomme comme ça? En fait j’ai aucune idée honnêtement si tous les agents que je vient de nommer fonctionnent avec des algos classiques ou pas.]

  Dans cet article, nous tenterons de répondre à chacune de ses questions en se basant sur les parutions d'études récentes. Pour ce faire, nous déquortiquerons le problème en plusieurs sous-étapes qui devront être accomplies par notre agent conversationnel hypothétique. De cette façon, il sera beaucoup plus simple d'avoir une représentation mentale des différentes parties d’une architecture entièrement neurale pour les agents conversationnels.
\end{abstract}
