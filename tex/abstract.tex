\begin{abstract}
  Les approches par réseaux de neurones ont récemment surpassées les approches par algorithmes classiques pour ce qui en est des agents conversationnels, tels que Siri et Alexa, Google Assistant, Google Home
  %[sources? Et est-ce que google assistant ca ce nomme comme ça? En fait j’ai aucune idée honnêtement si tous les agents que je vient de nommer fonctionnent avec des algos classiques ou pas.]
  Plusieurs techniques sont apparues, ce qui cause……? Dans cet article, un survol des différentes techniques existantes est fait, de façon à ce que le lecteur ait une idée des différentes parties d’une architecture entièrement neurale pour les agents conversationnels.
\end{abstract}
