\begin{abstract}
  Les approches par réseaux de neurones ont récemment surpassées les approches par algorithmes classiques pour la majorité des problèmes du traitement de la langue naturelle lorsqu'assez de données sont disponibles. C'est principalement ce qui explique pourquoi nous voyons de plus en plus de solutions commerciales implémentant des agents conversationnels, tels que \textbf{Siri (Apple)}\footnote{\url{https://www.apple.com/ca/ios/siri/}}, \textbf{Alexa (Amazon)}\footnote{\url{https://developer.amazon.com/alexa}} et \textbf{Google Assistant(Google)}\footnote{\url{https://assistant.google.com/}}. Alors, comment créer son propre agent conversationnel à partir de publications récentes et de technologies de pointe? Ainsi, nous regroupons et expliquons tous les sous-systèmes nécessaires à la construction d'un tel agent qui sera en mesure de recueillir une commande vocale pour ensuite renvoyer une réponse provenant d'une recherche dans une base de données de connaissances en texte naturel, telle que \textbf{Wikipédia}. Un aspect intéressant de la solution proposée est que la majorité des algorithmes exploités se basent sur les réseaux de neurones artificiels, regroupant ainsi une base de connaissances commune à chacune des étapes du traitement facilitant par le fait même l'intégration et le maintien d'une telle architecture. Dans des travaux ultérieurs, une telle architecture pourrait éventuellement être intégrée en un seul gros réseau de neurones profonds bout à bout plutôt qu'en plusieurs réseaux distincts juxtaposés les uns aux autres.
\end{abstract}
