\begin{abstract}
  Les approches par réseaux de neurones ont récemment surpassées les approches par algorithmes classiques pour la majorité des problèmes du traitement de la langue naturelle lorsqu'assez de données sont disponibles. C'est principalement ce qui explique pourquoi nous voyons de plus en plus de solutions commerciales implémentant des agents conversationnels, tels que \textbf{Siri (Apple)}\footnote{\url{https://www.apple.com/ca/ios/siri/}}, \textbf{Alexa (Amazon)}\footnote{\url{https://developer.amazon.com/alexa}} et \textbf{Google Assistant(Google)}\footnote{\url{https://assistant.google.com/}}. Bien que ces systèmes sont commerciaux, il doit exister des connaissances publiques permettant d'en venir à de telles architectures conversationnelles, et il est de l'intérêt du domaine public d'en faire un survol et d'y mettre de la lumière. Nous nous demandons de comment créer son propre agent conversationnel à partir de publications publiques et de technologies open-source de pointe, toujours en se basant sur les parutions d'études récentes. Ainsi, nous regroupons et expliquons tous les algorithmes nécessaires à la construction d'un tel agent artificiel, où il est possible de formuler une question par commande vocale, pour en obtenir une réponse suite à une recherche dans une base de données de connaissances en texte naturel, telle que Wikipédia. Pour ce faire, le problème est décortiqué en plusieurs étapes intermédiaires où il est possible d'utiliser plusieurs algorithmes différents à chaque étape. La beauté de la chose est que tous les algorithmes ici à l'étude se basent sur les réseaux de neurones artificiels, regroupant ainsi une base de connaissance commune à chacune des partie d'un pipeline. Cela facilite l'intégration d'une telle architecture. Dans des travaux ultérieurs, une telle architecture pourrait éventuellement être intégrée en un seul gros réseaux de neurones profond bout-à-bout plutôt qu'en plusieurs réseaux de neurones profonds collés ensemble. 
\end{abstract}
