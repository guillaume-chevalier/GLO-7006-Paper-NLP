\section{Développement}

Toutes les parties nécessaires pour concevoir un agent conversationnel basé sur une architecture neuronale existent présentement. Certaines techniques se sont démarquées au fil des études. Un survol rapide des techniques les plus prometteuses est fait, de façon à ce qu'il soit possible de joindre toutes ces techniques ensemble afin de créer un seul agent conversationnel complet, ce qui permettrait d'en faire une implémentation réelle complète. Les approches neuronales les plus aux goûts du jour sont à favoriser et sont introduites dans cet article, ces approches imitent la nature et sont déjà la forme la plus répandue d'intelligence (les réseaux de neurones), il ne reste qu'à les informatiser convenablement et à découvrir les bonnes configurations neuronales, un focus au niveau des algorithmes et de la logique dans le but de créer l'interface conversationnelle idéale. \\

\subsection{Traitement d'un intrant vocal}
La première étape de calcul au sein d’une architecture neurale destinée à comprendre et répondre à un utilisateur est de comprendre ce qu’il dit. Pour accomplir cette tâche, il est possible d’utiliser un réseaux de neurones TC-DNN-BLSTM-DNN, c’est-à-dire, des convolutions temporelles (TC) suivies de couches de neurones linéaires profondes (DNN), d’un LSTM Bidirectionnel (BLSTM) et puis d’un second DNN final \cite{acousticModeling}. Ainsi, cette architecture dépend d’un pré-traitement du signal par un autre algorithme lequel est plus classique et permet de transformer le signal en un domaine de fréquences personnalisées. C’est ce pré-traitement de l’information qui est introduit dans le réseau de neurones profond, afin d’en analyser le sens et de pouvoir convertir cela en états acoustiques, lesquels peuvent être convertis, cette fois, en texte littéraire. Cette architecture neurale, imagée à la \autoref{fig:tcDnnBlstmDnn}, obtient un WER (Word Error Rate) de retranscription de 3.47, ce qui est présentement l’état de l’art (SOTA) sur le jeux de données et problème du Wall Street Journal (WSJ) eval’92 \\%[PAPER REQUIS POUR LE WSJ eval’92, CITATION].

\begin{figure*}
  \centering
  \includegraphics[width=\textwidth]{tcDnnBlstmDnn}
  \caption{Architecture neurale TC-DNN-BLSTM-DNN [rajouter reference ici]}
  \label{fig:tcDnnBlstmDnn}
\end{figure*}

L’architecture neurale TC-DNN-BLSTM-DNN permet d’écouter le signal audio à l’aide des données audio extraites en fMMLR. Ainsi, un DNN suivi d’un BLSTM peut analyser ce signal pour classifier cela en états acoustiques, lesquels sont eux-mêmes repris par un algorithme classique qui permet de rassembler ces états en mots réels [DeepRecurrentNeuralNetworksForAcousticModelling]. Notons que cette architecture neural peut être utilisée pour raffiner le signal des mots prononcés, ce qui peut être envoyé directement dans un réseaux de neurones supérieur en tant que plongeage.

\subsection{Extraction des composantes de l'intrant et des sources d'information à analyser}
Une fois que la requête de l'utilisateur aura été convertie sous une forme textuelle facilement manipulable par un ordinateur, nous pourrions, dès lors, utiliser le plongement induit par l'étape précédente. Une autre approche consiste à reprendre cette sortie pour ensuite la fournir à une nouvelle structure qui se chargera d'aller extraire de nouvelles composantes qui aideront certainement à obtenir de meilleurs résultats pour la suite du processus. \\

À ce stade, nous devons comprendre que le signal est encore purement textuel et nous n'avons pour seule information qu'une décomposition des mots qui forment la demande reçue. Cependant, les langages sont formés de davantage de subtilités qu'un simple enchaînement de mots les uns après les autres. En effet, chaque mot joue un rôle précis dans la structure de la phrase et apporte une nuance particulière au contexte générale de celle-ci ou encore du texte avec une plus faible portée. C'est exactement ce que les travaux de ... visaient à faire. Ainsi, en ..., ce groupe de chercheurs a fait la publication d'un article détaillant leur approche en comparant plusieurs modèles différents comprenant autant des approches classiques que des approches neuronales. En plus de faire état de leurs travaux, ce groupe est aussi à l'origine d'outil qui est encore à ce jour considéré comme un incontournable: \cite{word2vec}. \\

Malgré le fait que cet article porte sur les approches neuronales, cet outil a plutôt prouvé que des approches plus simplistes et classiques sont parfois plus appropriées. Word2vec se fonde sur la combinaison de deux approches nommées \textit{Continuous Bag Of Words} (\autoref{fig:cbow}) et \textit{Skip-gram} (\autoref{fig:skipgram}). Alors que le \textit{Skip-gram} se concentre à essayer de prédire son contexte, le \textit{CBOW} cherchera plutôt à prédire la valeur considérée à partir de son environnement accordant ainsi plus d’importance à la structure des phrases plutôt qu’au contexte d’utilisation.\\

\begin{figure}[ht]
  \centering
  \includegraphics[width=\columnwidth, height=0.35\textheight, keepaspectratio]{cbow}
  \caption{Architecture de la méthode de prédiction CBOW [\citenum{word2vec}]}
  \label{fig:cbow}
\end{figure}

\begin{figure}[ht]
  \centering
  \includegraphics[width=\columnwidth, height=0.35\textheight, keepaspectratio]{skipgram}
  \caption{Architecture de la méthode de prédiction Skip-gram [\citenum{word2vec}]}
  \label{fig:skipgram}
\end{figure}

En fournissant la requête reçue à cet outil, il sera donc possible d'extraire les composantes sémantiques et syntaxiques sous-entendues par cette dernière. Par la suite, ces nouvelles composantes seront combinées à celle que nous avions déjà obtenues à l'étape précédente. En procédant avec cette seconde approche, nous réaliserons certainement un gain majeur au niveau de la performance des prochaines étapes en raison de l'ajout important de dimensionnalités qui fourniront beaucoup plus de flexibilité aux réseaux de neurones suivantes qui devront à leur tour détecter les nuances du langage. À titre d'exemple, lorsqu'un utilisateur demandera à l'assistant si ce dernier peut lui indiquer l'horaire du cinéma le plus prêt de sa position, l'assistant devra comprendre la nuance que ce qui intéresse vraiment l'utilisateur est l'horaire et non pas l'évaluation booléenne de sa capacité à s'acquitter de cette tâche. Par contre, dans le cas où l'utilisateur demanderait à l'assistant si ce dernier peut le connecté à l'Internet, l'assistant devra dans ce cas faire l'évaluation de sa capacité et répondre en par affirmation à notre cher utilisateur. \\
%TODO Référence au transfer learning

Mais qu'en est-il de nos sources d'informations? En fait, le processus entier bénéficiera certainement qu'un travail similaire soit fait à ce niveau aussi. Pour ce faire, deux approches s'offrent encore à nous. La première consistant encore une fois à utiliser word2vec et la seconde repose sur le même principe, mais à un niveau supérieur d'abstraction en considérant cette fois l'utilité de chacune des phrases dans le texte plutôt que de se concentrer sur le rôle de chaque mot dans chaque phrase \cite{inferSent}. \\

En somme, toutes ces composantes ainsi dérivée pourrons ensuite être fournies en entrée d'un réseau de neurones tel qu'un RNN tel qu'il sera expliqué à section suivante.

\subsection{Interprêter la requête et cibler le contenu d'intérêt pour y répondre}

Pour analyser les demandes de l'utilisateur, il faut les traduire en requêtes neurales pour ensuite rechercher dans le texte les passages intéressants. Cela est une étape importante à comprendre avant d'analyser d'avantage ce qui sera expliqué dans les prochaines sections.

1. Mécanismes d’attention: expliquer et introduire les papers. Faire ca dans l’intro??

2. QA pur et dur

\subsection{Formulation d'une réponse à partir de l'information d'intérêt retenue}

Bien qu’il est intéressant de trouver l’endroit où porter attention dans un corpus textuel, il est tout autant intéressant de savoir comment générer une réponse structurée et concise à l’utilisateur. Cela peut être fait en utilisant le \gls{hred} tel qu’introduit par Iulian V. Serban et al. \cite{chatbot\string:HRED}. En effet, HRED est une imbrication hiérarchique de réseaux de neurones récurrents. Un premier est utilisé afin d’encoder les phrases, un second est nécessaire afin de garder le contexte des réponses passées lesquelles ont déjà traitées, comme un suivi de la discussion dans une mémoire temporaire, et finalement un troisième RNN est mis à profit afin de décoder l’information en une réponse à l’utilisateur. En adaptant l'architecture neurale \gls{hred} de façon à lui donner des mécanismes d'attention tels que précédemment expliqués, il est possible de générer la réponse en retour de la requête attentionnelle à l’utilisateur. Ainsi, en ayant le contexte de la question que l'utilisateur pose ainsi que le contexte des documents à parcourir avec les mécanismes d’attention, il est possible de chercher dans le texte ce qu'il faut comme information, pour faire un calcul sur cela, ce qui est envoyé au décodeur du \gls{hred} lequel peut répondre avec le nouveau contexte de l'information trouvée par la recherche effectuée. Ainsi, le premier RNN du HRED qui encode l’information peut utiliser word2vec \cite{word2vec} directement, en plus d’utiliser un \textit{embedding} provenant de l’avant dernière couche de neurones du DNN (Deep Neural Network) de STT (Speech to Text). En plus de cela, il est possible d’utiliser le réseau de neurones infersent de \textbf{Facebook} \cite{inferSent}, lequel peut être concaténé au signal de sortie du RNN encodeur du HRED, en tant que plongement supplémentaire au niveau des phrases plutôt qu’au niveau des mots. \\

Dans une amélioration plus récente de l’architecture HRED \cite{chatbot\string:LVHRED}, il est possible d’utiliser une variable latente intermédiaire laquelle permet de faire le pont entre les réponses envoyées du décodeur vers l’utilisateur, en plus de réinjécter cette réponse dans l’encodeur qui écoute la réponse de l’utilisateur suite à cela. En retournant ainsi l'information du du décodeur dans l’encodeur, nous nous assurons de conserver le contexte d’une phrase à la prochaine et d'ainsi avoir un discours plus fluide tout en étant moins assujettis à des variations subites de sujet ou d'interprétation. D'autre part, ce passage d'information aura pour effet de renforcer la qualité de la requête attentionnelle laquelle peut être générée à la toute fin de l’encodeur du HRED. C'est à ce moment que le mécanisme d’attention décrit dans la section précédente portant sur l’analyse de texte suite à des questions pourrait être inséré. Une fois la question posée par l’utilisateur et lue dans l’encodeur, le HRED peut bénéficier de cette question dans son RNN intermédiaire, et ce, en tant que requête attentionnelle à passer directement au système attentionnel. Des travaux similaires ont été réalisés par Karl Moritz Hermann et al. chez \textbf{Google} \cite{readNcomprehend}, lesquels sont ici inspirants. Somme toutes, le HRED aura accès à la question de l’utilisateur et au corpus de texte dans lequel il peut maintenant cibler l’information pertinente.

Étant donné la taille énorme du corpus textuel dans lequel le réseaux de neurones peut lire l’information, tel que l’ensemble du texte sur Wikipédia par exemple, il est possible d’appliquer un MapReduce \cite{DeanMapReduce} pour ainsi améliorer les performances de ce processus et ainsi de façon importante le temps de réponse de notre assistant, ce qui est un aspect primordial. Cette technique procède de façon distribuée sur plusieurs centaines d’ordinateurs. Dans le cas présent, ceux-ci utiliseraient eux-même les implémentations de word2vec et infersent sur le corpus d'information, et cela en ayant en main la requête attentionnelle générée par le mécanisme d'attention, selon les principes de MapReduce. Cette partie, qui est distribuée et qui est surnommée, le lecteur impatient, est représenté à la \autoref{fig:teachingImpatientReader}. Il y a même une amélioration possible sur cet architecture neurale. Il est visible dans la figure que plusieurs itérations entre la requête et le système attentionnel est fait. Cela devrait être fait en une seule étape afin de réduire la complexité algorithmique de linéaire à constante en fonction de la longueur de la requête, en termes de nombre de mots. La recherche suite à la requête pouvant être distribuée, cela peut être fait en un temps très rapide, tout comme l'ensemble des opérations décrites dans les sections précédentes.

\begin{figure*}
  \centering
  \includegraphics[width=\textwidth]{teachingImpatientReader}
  \caption{Le lecteur impatient prends la requête “X visited England” afin de faire une recherche dans le texte “Mary went to England”, à l’aide du mécanisme d’attention lequel est ici dénoté “r”, assisté de la requête “u” [rajouter reference ici]}
  \label{fig:teachingImpatientReader}
\end{figure*}

\subsection{Retourner la réponse textuelle sous la forme d'un signal audio}
Une fois une réponse générée, il est intéressant de générer l’audio de cette à nouveau afin de répondre à l’utilisateur, ce qui est un autre calcul rapide qui peut se faire en temps réel. Cela est possible avec le CNN (Conv.. neural net) Wavenet d’Aaron van den Oord et al., développé chez Google \cite{wavenet}. En effet, il est possible de générer n’importe quel ton de voix avec \texttt{Wavenet}, ainsi le choix de la voix de la personne qui parle peut être fait par l’utilisateur. À titre d’exemple, cette architecture neurale est tellement puissante qu’il est possible de lui faire imiter la voix du président. Cette découverte récente par Google est la première fois qu’il est possible de confondre la voix pour une voix humaine réelle plutôt qu’une voix robotique, ainsi l’illusion est bien réussie. La façon dont \texttt{Wavenet} fonctionne est d’établir un préalable statistique (une variable conditionnée) qui est donnée à un premier algorithme qui s’occupe de trouver les bons tons de voix à générer avec \texttt{Wavenet}, à partir du texte. C’est ainsi que Wavenet, conditionné lui-même par le ton de voix demandé et par le texte, peut générer la voix de façon réalistique. C’est une méthode point par point, ainsi, chaque point dans la vague audio est généré en fonction des points précédents et du conditionnement demandé, c’est très bas niveau sur le signal qui est à un taux d’échantillonage de 16 kHz lors de l’entraînement, ce qui est assez pour capturer les subtilités de quelqu’un qui parlerait réellement dans un enregistrement. Cette phase générative est illustrée dans dans la \autoref{fig:wavenet}.

\begin{figure*}
  \centering
  \includegraphics[width=\textwidth]{wavenet}
  \caption{À l’aide de convolutions causales dilatées, il est possible de prédire le prochain point dans la vague audio de façon efficace. Cela est un modèle autorégréssif : les points passés sont utilisés pour prédire les points suivants du même signal. Ainsi, la sortie est remise en entrée pour le calcul de l’étape suivante, ce sampling peut faire usage de mémoire cache, ce qui donne à cet algorithme générationnel un temps linéaire pour la génération, cela en fonction de la longueur du signal à générer \cite{wavenet}.}
  \label{fig:wavenet}
\end{figure*}

