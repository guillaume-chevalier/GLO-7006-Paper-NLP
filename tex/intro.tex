\section{Introduction}
Depuis que la naissance de l'informatique, l'humain a toujours convoité l'idée de pouvoir intéragir verbalement avec un ordinateur, et ce, de manière totalement transparente comme s'il s'agissait d'un autre humain apte de capter la majorité des nuances du discours qu'il entretiendra. Bien que ce sujet aura fait couler beaucoup d'encre et fait tourner les têtes, nous ne connaissons toujours pas à ce jour une formule secrète pour parvenir à sa réalisation. Au cours des années, différentes démarches ont été proposées. Traditionnellement, des approches algorithmiques étaient favorisées et certains projets se fondent encore sur ces dernières, tel que \textbf{Watson} de \textbf{IBM}. Ces approches ont toutefois le défaut d'être longues et ardues à développer et la réutilisation des travaux sous-jacents n'est qu'encore plus complexe en raison du caractère très sur mesure du problème ou du champ d'application auquel il s'applique. \\
%EDIT: je parlerais d’IBM Watson, mais c’est une approche classique et c’est pas neuronal. C’est du 2014-2015, thought, donc ça se placerait bien dans l’intro. Faudrait trouver un de leurs papers ici, mais ça paraît que toute leur shit c’est des algos classiques, c’est décevant de leur part: http://researcher.watson.ibm.com/researcher/view_group_pubs.php?grp=2099

Récemment, des approches mettant en jeu des réseaux de neurones artificiels ont leur apparition et ont immédiatement connus beaucoup de succès. À titre d'exemple, en 2014, un grand pas a été réalisé est fait lorsque les techniques par réseaux de neurones viennent à dépasser les performances des techniques classiques pour la tâche de faire de la traduction automatique \cite{attentionMechanism}. C’est aussi de tels systèmes qui sont désormais utilisés chez \textbf{Google} pour la mise en production du fameux \textbf{Google Translate} \cite{googleTranslate}. Cette même compagnie utilise aussi des algorithmes de \textit{Speech-to-Text} afin de pouvoir générer des sous-titres automatiquement pour les vidéos \textbf{YouTube} et afin de pouvoir analyser les vidéos et les lier entre elles avec une approche sémantique.
%TODO: TROUVER UN PAPER ICI: https://research.google.com/pubs/SpeechProcessing.html.
D’ailleurs, il est dorénavant possible de générer de l’audio en temps réel avec une approche par réseaux de neurones constitutionnels \cite{wavenet}. \\

Il ne s'agit ici que de différents morceau de puzzle qui mèneront éventuellement à la création d'un agent conversationnel complet. Cet notamment ce qui explique pourquoi la création d'un tel agent est une tâche aussi compliquée. Dans le cadre d'un échange verbal entre deux êtres, une multitude de tâches sont accomplies sans même que nous ne soyons conscient. Le tout débute lors d'un contact initial le plus souvent dans une forme auditive vers un destinataire. En partant de ce point, à titre de destinataire, nous devrons premièrement capter ce message, malgré des obstacles environnants réduisant la qualité de ce dernier, filtrer ce qui est réellement important dans le signal et le décoder selon un dialecte sous-entendu par l'emplacement sur le globe terrestre où nous nous trouvons. Une fois en possession de ce message, nous établirons des liens entre l'énoncé qui a été donné et le registre de connaissances que nous possèdons. Nous établirons ensuite quelle est la réponse la plus appropriée compte tenu d'une panoplie de facteurs comme l'identité de notre interlocuteur, nos valeurs, nos connaissances, etc. \\

Une fois avec cette réponse en main, nous ne sommes rendus qu'à la moitié du parcours puisque nous devrons refaire la totalité de ce trajet à l'inverse. Ainsi, nous structurerons la réponse ainsi trouvée sous une forme syntaxique et sémantique suffisamment complète afin de favoriser une compréhension immédiate en mettant à profit notre vocabulaire et les différentes règles qui définissent une utilisation adéquate du dialecte sous lequel l'échange a été initiée. Finalement, nous émettrons à notre tour ce signal vers notre interlocuteur. Bien entendu, il ne s'agit que du chemin traditionnel d'une conversation, mais il serait simple d'y intégrer les nombres processus de traductions qui sont aussi mis en jeu lorsque les deux personnes ne possède pas la même langue maternelle compliquant davantage l'ensemble du processus. Pour rajouter encore plus de difficulté, nous devrons répéter toutes ces étapes dans un interval de temps très rapide pour éviter que la conversation devienne simplement impossible à suivre ou encore que notre interlocuteur se lasse de cet échange et préfère ainsi l'interrompre.\\

Dans cet article, nous aborderons ainsi chacune de ses étapes en précisant comment nous pouvons parvenir à les résoudres en favorisant plus souvent qu'autrement les approches neuronales les plus aux goûts du jour.
